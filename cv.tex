\documentclass[hidelinks]{article}

\usepackage{xcolor}
\usepackage{fontawesome}
\usepackage[misc]{ifsym}
\usepackage{tikz}
\usepackage[left=0.5cm,top=3.1cm,right=1cm,bottom=1cm]{geometry}
\usepackage{fontspec}
\usepackage{calc}
\usepackage{fontawesome}
\usepackage{tabularx}
\usepackage{sectsty}
\usepackage{booktabs}
\usepackage{graphicx}
\usepackage{float}

\usepackage{hyperref}

\tolerance=1
\emergencystretch=\maxdimen
\hyphenpenalty=10000
\hbadness=10000

\definecolor{mediumgray}{HTML}{606060}
\definecolor{lightgray}{HTML}{999999}
\definecolor{gray}{HTML}{4D4D4D}
\definecolor{green}{HTML}{a4b365}
\definecolor{purple}{HTML}{a568d3}
\definecolor{blue}{HTML}{2eb8b8}
\definecolor{date}{HTML}{A4A4A4}
\definecolor{pink}{HTML}{F778A1}
\definecolor{orange}{HTML}{e7b382}

\colorlet{fillheader}{gray}
\colorlet{headertext}{white}

\def\printWhiteHeader{0}
\def\printGrayHeader{0}
\def\pink{0}
\def\blue{0}

\if\blue1
    \colorlet{orange}{blue}
    \colorlet{green}{blue}
    \colorlet{purple}{blue}
    \colorlet{pink}{blue}
\fi
\if\printWhiteHeader1
    \colorlet{lightgray}{mediumgray}
    \colorlet{orange}{mediumgray}
    \colorlet{green}{mediumgray}
    \colorlet{purple}{mediumgray}
    \colorlet{blue}{mediumgray}
    \colorlet{pink}{mediumgray}
    \colorlet{fillheader}{white}
    \colorlet{headertext}{gray}
\fi
\if\printGrayHeader1
    \colorlet{lightgray}{mediumgray}
    \colorlet{orange}{mediumgray}
    \colorlet{green}{mediumgray}
    \colorlet{purple}{mediumgray}
    \colorlet{blue}{mediumgray}
    \colorlet{pink}{mediumgray}
    \colorlet{fillheader}{gray}
    \colorlet{headertext}{white}
\fi

\newfontfamily\bodyfont{Roboto-Regular}[Path=fonts/]
\newfontfamily\bodyfontit{Roboto-LightItalic}[Path=fonts/]
\newfontfamily\thinfont{Roboto-Thin}[Path=fonts/]
\newfontfamily\headingfont{RobotoCondensed-Bold}[Path=fonts/]

\defaultfontfeatures{Mapping=tex-text}
\setmainfont[Mapping=tex-text, Path = fonts/]{Roboto-Light}

\sectionfont{\large \headingfont}

\color{gray}

\begin{document}

\pagestyle{empty}

\begin{tikzpicture}[remember picture,overlay]
  \node [rectangle, fill=fillheader, anchor=north, minimum width=\paperwidth, minimum height=3cm] (box) at (current page.north){};
  \node [anchor=center] (name) at (box) {
    \fontsize{40pt}{65pt}\color{headertext}{\thinfont Helena Cruz}};
\end{tikzpicture}

\begin{minipage}[t]{4.6cm}
    \vspace*{0pt}
    \section*{\textcolor{blue}{About me}}

        \includegraphics[scale=0.32]{hel}

        \vspace{0.3cm}

        \begin{tabularx}{4.6cm}{@{}p{0.1cm}l}
            \textcolor{gray}{\Letter} & \href{mailto:hsrcruz@gmail.com}{{hsrcruz@gmail.com} } \\
            \textcolor{gray}{\faPhone} & +351 964 369 693 \\
            \textcolor{gray}{\faGithub} & \href{https://github.com/helenacruz}{helenacruz} \\
            \textcolor{gray}{\faLinkedin} & \href{https://linkedin.com/helcruz}{helcruz} \\
        \end{tabularx}
%
    \section*{\textcolor{orange}{Programming Languages}}
        C, C++, Java, Python
%
    \section*{\textcolor{pink}{Tools and Technologies}}
        Vim, Git, IntelliJ, Visual Studio, Xilinx SDK, Arduino, Linux
%
    \section*{\textcolor{green}{Languages}}
        
        \ifnum\ifnum\printGrayHeader=1 1\else\ifnum\printWhiteHeader=1 1\else0\fi\fi
        = 1
            {\small native} \\
        \else
            \textcolor{lightgray}{native} \\
        \fi
        \null \quad Portuguese

        \ifnum\ifnum\printGrayHeader=1 1\else\ifnum\printWhiteHeader=1 1\else0\fi\fi
        = 1
            {\small fluent} \\
        \else
            \textcolor{lightgray}{fluent} \\
        \fi
        \null \quad English

        \ifnum\ifnum\printGrayHeader=1 1\else\ifnum\printWhiteHeader=1 1\else0\fi\fi
        = 1
            {\small beginner} \\
        \else
            \textcolor{lightgray}{beginner} \\
        \fi
        \null \quad Spanish
\end{minipage}\hfill
%
\hspace{0.6cm}
%
\begin{minipage}[t]{\textwidth - 6.1cm}
    \vspace*{0pt}
    \section*{\textcolor{pink}{Education}}
        \begin{tabularx}{\linewidth}{@{}lXc}
            2016--2018 & {\headingfont MSc. Information Systems and Computer Engineering} & \begin{tabular}[c]{@{}c@{}}\small \textcolor{lightgray}{Instituto Superior Técnico}\\ \small \textcolor{lightgray}{University of Lisbon}\end{tabular} \\
                      & \multicolumn{2}{p{11.75cm}}{\textcolor{lightgray}{\footnotesize Coursework:} \footnotesize Applications for Embedded Systems; Networked Embedded Systems; Ambient Intelligence;
                      Software Security;  Parallel and Distributed Computation; Robotics; Programming Languages.} \\[0.6cm]

            2013--2016 & {\headingfont BSc. Information Systems and Computer Engineering} & \begin{tabular}[c]{@{}c@{}}\small \textcolor{lightgray}{Instituto Superior Técnico}\\ \small \textcolor{lightgray}{University of Lisbon}\end{tabular} \\
                & \multicolumn{2}{p{11.75cm}}{\textcolor{lightgray}{\footnotesize Coursework:} \footnotesize Operating Systems; Software Engineering; Databases; Algorithms; Compilers; 
                Data Structures; Computer Architecture; Distributed Systems; Artificial Intelligence; Object-Oriented Programming.} \\[0.6cm]

            2010--2013 & {\headingfont High School Science and Technologies} & \begin{tabular}[c]{@{}c@{}}\small \textcolor{lightgray}{Escola Secundária}\\ \small \textcolor{lightgray}{de S. Lourenço}\end{tabular} \\

            \end{tabularx}

    \section*{\textcolor{purple}{Master Thesis}}

    \small During my master's, I've developed an on-board multi-core and fault-tolerant embedded architecture for Synthetic-Aperture Radar (SAR)
    image generation systems. The architecture was developed considering the space environment, which 
    can cause bit-flips due to the radiation.

    \smallskip

    \small The developed architecture implemented two fault tolerance mechanisms: lockstep
    and reduced-precision redundancy and aims to protect the Backprojection algorithm from transient faults, using a
    software-only approach, generating acceptable SAR images in a space environment. 
    The solution was implemented using a SoC, a Zybo board from Digilent, with a Zynq device, 
    containing a dual-core ARM Cortex-A9 processor and a Xilinx FPGA.    
    The architecture was developed in C and the Xilinx SDK was used. 

    \section*{\textcolor{blue}{Selected Projects}}

    \begin{tabularx}{\linewidth}{@{}Xr}
        {\headingfont Compiler} & \textcolor{lightgray}{\small \href{https://github.com/helenacruz/zu-compiler}{github.com/helenacruz/zu-compiler}} \\
        \multicolumn{2}{@{}p{\linewidth}}{\small Implementation of a compiler for a simple language called Zu in C++, using Yacc and Lex.} \\%[0.3cm]
        \addlinespace[0.3cm]
        {\headingfont MBED nodes with CAN} & \textcolor{lightgray}{\small \href{https://github.com/helenacruz/Interconnecting-mbed-nodes-CAN}{github.com/helenacruz/Interconnecting-mbed-nodes-CAN}} \\
        \multicolumn{2}{@{}p{\linewidth}}{\small
            Implementation of a CAN (Controller Area Network) using MBED nodes, the FreeRTOS operating system and several additional sensors such as flame, ultrassonic and temperature sensor.
        } \\%[0.6cm]
        \addlinespace[0.3cm]
        {\headingfont Parallel 3D Game of Life} & \textcolor{lightgray}{\small \href{https://github.com/helenacruz/game-of-life3d}{github.com/helenacruz/game-of-life3d}} \\
        \multicolumn{2}{@{}p{\linewidth}}{\small Implementation of a parallel 3D Game of Life in C++ using OpenMP and MPI.} 
    \end{tabularx}

    \section*{\textcolor{orange}{Interests}}

        \small My main interests include embedded systems, critical systems, fault tolerance and
        low-level programming. I also like to play around with boards and sensors.

        \smallskip

        \small When I'm not at a computer, I'm probably reading a book, watching a TV show or 
        annoying my cat.

\end{minipage}\hfill

\end{document}